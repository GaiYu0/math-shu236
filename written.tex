\documentclass{article}

\usepackage{amsmath}
\usepackage{amssymb}
\usepackage{amsthm}
\usepackage{dsfont}
\usepackage{commath}
\usepackage{amsfonts}
\usepackage{bm}

\newcommand{\bba}{{\mathbb A}}
\newcommand{\bbb}{{\mathbb B}}
\newcommand{\bbc}{{\mathbb C}}
\newcommand{\bbd}{{\mathbb D}}
\newcommand{\bbe}{{\mathbb E}}
\newcommand{\bbf}{{\mathbb F}}
\newcommand{\bbg}{{\mathbb G}}
\newcommand{\bbh}{{\mathbb H}}
\newcommand{\bbi}{{\mathbb I}}
\newcommand{\bbj}{{\mathbb J}}
\newcommand{\bbk}{{\mathbb K}}
\newcommand{\bbl}{{\mathbb L}}
\newcommand{\bbm}{{\mathbb M}}
\newcommand{\bbn}{{\mathbb N}}
\newcommand{\bbo}{{\mathbb O}}
\newcommand{\bbp}{{\mathbb P}}
\newcommand{\bbq}{{\mathbb Q}}
\newcommand{\bbr}{{\mathbb R}}
\newcommand{\bbs}{{\mathbb S}}
\newcommand{\bbt}{{\mathbb T}}
\newcommand{\bbu}{{\mathbb U}}
\newcommand{\bbv}{{\mathbb V}}
\newcommand{\bbw}{{\mathbb W}}
\newcommand{\bbx}{{\mathbb X}}
\newcommand{\bby}{{\mathbb Y}}
\newcommand{\bbz}{{\mathbb Z}}

\newcommand{\bma}{{\bm a}}
\newcommand{\bmb}{{\bm b}}
\newcommand{\bmc}{{\bm c}}
\newcommand{\bmd}{{\bm d}}
\newcommand{\bme}{{\bm e}}
\newcommand{\bmf}{{\bm f}}
\newcommand{\bmg}{{\bm g}}
\newcommand{\bmh}{{\bm h}}
\newcommand{\bmi}{{\bm i}}
\newcommand{\bmj}{{\bm j}}
\newcommand{\bmk}{{\bm k}}
\newcommand{\bml}{{\bm l}}
\newcommand{\bmm}{{\bm m}}
\newcommand{\bmn}{{\bm n}}
\newcommand{\bmo}{{\bm o}}
\newcommand{\bmp}{{\bm p}}
\newcommand{\bmq}{{\bm q}}
\newcommand{\bmr}{{\bm r}}
\newcommand{\bms}{{\bm s}}
\newcommand{\bmt}{{\bm t}}
\newcommand{\bmu}{{\bm u}}
\newcommand{\bmv}{{\bm v}}
\newcommand{\bmw}{{\bm w}}
\newcommand{\bmx}{{\bm x}}
\newcommand{\bmy}{{\bm y}}
\newcommand{\bmz}{{\bm z}}

\newcommand{\bmA}{{\bm A}}
\newcommand{\bmB}{{\bm b}}
\newcommand{\bmC}{{\bm c}}
\newcommand{\bmD}{{\bm d}}
\newcommand{\bmE}{{\bm e}}
\newcommand{\bmF}{{\bm f}}
\newcommand{\bmG}{{\bm g}}
\newcommand{\bmH}{{\bm h}}
\newcommand{\bmI}{{\bm i}}
\newcommand{\bmJ}{{\bm j}}
\newcommand{\bmK}{{\bm k}}
\newcommand{\bmL}{{\bm l}}
\newcommand{\bmM}{{\bm m}}
\newcommand{\bmN}{{\bm n}}
\newcommand{\bmO}{{\bm o}}
\newcommand{\bmP}{{\bm p}}
\newcommand{\bmQ}{{\bm q}}
\newcommand{\bmR}{{\bm r}}
\newcommand{\bmS}{{\bm s}}
\newcommand{\bmT}{{\bm t}}
\newcommand{\bmU}{{\bm u}}
\newcommand{\bmV}{{\bm v}}
\newcommand{\bmW}{{\bm w}}
\newcommand{\bmX}{{\bm x}}
\newcommand{\bmY}{{\bm y}}
\newcommand{\bmZ}{{\bm z}}

\newcommand{\cala}{{\mathcal A}}
\newcommand{\calb}{{\mathcal B}}
\newcommand{\calc}{{\mathcal C}}
\newcommand{\cald}{{\mathcal D}}
\newcommand{\cale}{{\mathcal E}}
\newcommand{\calf}{{\mathcal F}}
\newcommand{\calg}{{\mathcal G}}
\newcommand{\calh}{{\mathcal H}}
\newcommand{\cali}{{\mathcal I}}
\newcommand{\calj}{{\mathcal J}}
\newcommand{\calk}{{\mathcal K}}
\newcommand{\call}{{\mathcal L}}
\newcommand{\calm}{{\mathcal M}}
\newcommand{\caln}{{\mathcal N}}
\newcommand{\calo}{{\mathcal O}}
\newcommand{\calp}{{\mathcal P}}
\newcommand{\calq}{{\mathcal Q}}
\newcommand{\calr}{{\mathcal R}}
\newcommand{\cals}{{\mathcal S}}
\newcommand{\calt}{{\mathcal T}}
\newcommand{\calu}{{\mathcal U}}
\newcommand{\calv}{{\mathcal V}}
\newcommand{\calw}{{\mathcal W}}
\newcommand{\calx}{{\mathcal X}}
\newcommand{\caly}{{\mathcal Y}}
\newcommand{\calz}{{\mathcal Z}}


\title{Neural Networks: Theory and Practice}
\author{Yu Gai}
\date{}

\DeclareMathOperator{\deg}{deg}
\DeclareMathOperator{\diag}{diag}
\newcommand{\ind}{{\mathds 1}}
\newcommand{\pp}[2]{{\frac{\partial {#1}}{\partial {#2}}}}
\newtheorem{theorem}{Theorem}

\begin{document}

\maketitle

\tableofcontents

\newpage

\section{Multi-layer Perceptron}

\subsection{Forward and Backward Propagation}

Given an input $x \in \bbr^d$, the forward propagation rule of a $L$-layer perceptron with weights $\{W^{(l)}\}_{l = 1}^L$, biases $\{b_l\}_{l = 1}^L$, and activation function $\sigma: \bbr \mapsto \bbr$ is defined as
\begin{alignat*}{2}
h_0 & \triangleq x \\
\bar{h}_l & \triangleq W^{(l)} h_{l - 1} + b_l && \qquad l \in [L - 1] \\
h_l & \triangleq \sigma (\bar{h}_l) && \qquad l \in [L - 1] \\
h_L & \triangleq W^{(L)} h_{L - 1} + b_L
\end{alignat*}
where $[n] \triangleq \{1, ..., n\}$, $W^{(l)} \in \bbr^{d_l \times d_{l - 1}}$ with $d_0 \triangleq d$, $b_l \in \bbr^{d_l}$, and the application of $\sigma$ is pointwise.
Typical choices of $\sigma$ are the logistic function $\sigma (\cdot) = \frac1{1 + \exp (-\cdot)}$, the hyperbolic tangent $\sigma = \tanh$, and the rectified linear unit (ReLU) $\sigma (\cdot) = \max (0, \cdot)$.

Given a supervision signal $y \in \caly$ and a loss function $\ell: \bbr^{d_L} \times \caly \mapsto \bbr$, backward propagation refers to the computation of $\{W^{(l)}'\}_{l = 1}^L$ and $\{\pp{\ell}{b_l} (h_L, y)\}$, where $W_{i j}^{(l)}' \triangleq \pp{\ell}{W_{i j}^{(l)}} (h_L, y)$
it follows from the chain rule that the corresponding backward propagation rule is
\begin{alignat*}{2}
h_L' & \triangleq \nabla_{h_L} \ell (h_L, y) \\
W^{(L)}' & \triangleq \nabla_{W^{(L)}} \ell (h_L, y) = h_L' h_{L - 1}^T \\
b_L' & \triangleq \nabla_{b_L} \ell (h_L, y) = h_L' \\
h_{L - 1}' & \triangleq \nabla_{h_{L - 1}} \ell (h_L, y) = W^{(L) T} h_L' \\
\bar{h}_l' & \triangleq \nabla_{\bar{h}_l} \ell (h_L, y) = \diag (\sigma' (\bar{h}_l)) h_{l + 1}' && \qquad l \in [L - 1] \\
W^{(l)}' & \triangleq \nabla_{W^{(l)}} \ell (h_L, y) = \bar{h}_l' h_{l - 1}^T && \qquad l \in [L - 1] \\
b_l' & \triangleq \nabla_{b_l} \ell (h_L, y) = \bar{h}_l' && \qquad l \in [L - 1] \\
h_l' & \triangleq \nabla_{h_l} \ell (h_L, y) = W^{(l) T} \bar{h}_l' && \qquad l = 0, ..., L - 2
\end{alignat*}
where $\diag (x)_{i j} \triangleq x_i \delta_{i j}$ and the application of $\sigma'$ is again pointwise.

\subsection{Approximation theorems}

Let $I_d \triangleq [0, 1]^d$ and let $C (I_d)$ be the space of continuous functions on $I_d$.

\begin{theorem}[\cite{cybenko1989approximation}]
$2$-layer perceptrons with logistic activation function are dense in $C (I_d)$.
\end{theorem}

\begin{theorem}
$2$-layer perceptrons with logistic activation function are dense in $L^1 (I_d)$.
\end{theorem}

\begin{theorem}
$2$-layer perceptrons with logistic activation function are dense in $L^2 (I_d)$.
\end{theorem}

\subsection{The VC-dimension of multi-layer perceptrons}

\subsection{Empirical risk minimization with stochastic gradient descent}

\section{Neural networks for structured data}

Structured data refers to data that may be considered as a mapping $\Omega \mapsto \bbr^d$.
For example, in the case of sequential/temporal data, $\Omega$ is real intervals representing time;
in the case of image data, $\Omega$ is rectangles in $\bbr^2$;
in the case of graph data, $\Omega$ is a graph.

\subsection{Recurrent neural networks: neural networks for sequential/temporal data}

An example of sequential data is sentences in natural languages.
A natural motivation for the architecture of recurrent neural network (RNN) is the following.
Suppose we are trying to evaluate the probability for a sequence $\{x_t\}_{t = 1}^T \subset \bbr^d$ to occur.
It follows from the definition of conditional probability that
The motivation for RNN is to summarize $\{X_s = x_s : s < t\}$, i.e. the conditions at time $t$, by a vector $h_t \in \bbr^H$, and approximate.
The simplest
\[
W_x x + W
\]
\[
\bbp (X_1 = x_1, ..., X_T = x_T)
= \prod_{t = 1}^T \bbp (X_t = x_t | X_s = x_s\ \text{for}\ s < t)
\]
It is crucial that the number of parameters in recurrent neural networks is independent from the sequence length $T$, which enables RNN's to handle long sequences.
The performance of RNN depends significantly on how well the vector $h_t$ summarizes the conditions at time $t$.
It turns out that the simplest RNN can easily "forget" conditions in distant past.

\subsection{Principles of geometric deep learning}

a.k.a. geometric deep learning.

\subsection{Basics of harmonic analysis}

Given $f \in L^1 (\bbr^d)$, the Fourier transform of $f$ is defined as
\[
(\calf f) (\xi) = \int_{\bbr^d} e^{-i \xi x} f(x) \dif x
\]
It follows that $\calf f \in L^\infty (\bbr^d)$ for $f \in L^1 (\bbr^d)$ because
\begin{align*}
\sup_{\xi \in \bbr^d} |(\calf f) (\xi)|
& = \sup_{\xi \in \bbr^d} \left|\int_{\bbr^d} e^{-i \xi x} f(x) \dif x \right| \\
& \leq \sup_{\xi \in \bbr^d} \int_{\bbr^d} \left|e^{-i \xi x} f(x) \right| \dif x \\
& = \sup_{\xi \in \bbr^d} \int_{\bbr^d} |f(x)| \dif x \\
& = |f|_{L^1 (\bbr^d)}
\end{align*}
where $\sup$ denotes the essential supremum, as in the definition of $L^\infty (\bbr^d)$.

A fundamental property of the Fourier transform is that smoothness in the spatial domain implies decay in the frequency domain.
To state it in a general setting, we introduce the following definitions.
Given a $d$-dimensional multi-index $\alpha = (\alpha_1, ..., \alpha_d) \in \bbn^d$, the differential operator $D^\alpha$ is defined as
\[
D^\alpha \triangleq \frac{\partial^{\alpha_1}}{\partial x_1^{\alpha_1}} \cdot \ldots \cdot \frac{\partial^{\alpha_d}}{\partial x_d^{\alpha_d}}
\]
The order $|\alpha|$ of $\alpha$ is defined as
\[
|\alpha| \triangleq \sum_{i = 1}^d \alpha^i
\]
Define
\[
\calc^k \triangleq \{f : D^\alpha f\ \text{is continuous for all}\ |\alpha| \leq k\}
\]
We restrict $D^\alpha$ to $\calc^{|\alpha|}$, so that the ordering of partial differential operators in the definition of $D^{\alpha}$ does not matter by Schwarz's theorem.
For $1 \leq p \leq \infty$ and $k \in \bbn$, the ($L^p$)-Sobolev space of order $k$ is defined as
\[
W_p^k \triangleq \{f \in L^p : D^\alpha f \in L^p\ \text{for all}\ |\alpha| \leq k\}
\]
Stoke's theorem states that
\[
\int_\Omega \dif \omega = \int_{\partial \Omega} \omega
\]

The Laplacian is defined as
\[
\Delta \triangleq \sum_{i = 1}^d \frac{\partial^2}{\partial x_i^2}
\]
For a function $u: \bbr^{d + 1} \mapsto \bbr$, where the first $d$ dimensions are spatial and the last dimension is temporal, the heat equation is
\[
\pp{u}{t} = c \Delta u
\]

\subsection{Deep learning on images}

\subsubsection{Non-parametric convolutional neural networks}

It turns out that for some simple tasks, e.g. MNIST digit recognition, convolutional neural networks without learnable parameters can achieve state of the art performance as well.
In the following we describe the architecture of PCANet \cite{chan2015pcanet}, a convolutional neural network with filters inferred solely from Principle Component Analysis (PCA).
Consider a dataset consisting of $n$ image-label pairs
\[
\{(x_i, y_i)\}_{i \in [n]} \subset \bbr^{W H C} \times \caly
\]
where $W$, $H$, and $C$ are the width, height, and number of channels of images, and $\caly$ is the collection of labels.

\subsection{Deep learning on graphs}

In the following we describe neural networks tailored to data with graph structures.
More precisely, we are interested in data representable as $\calv \mapsto \bbr^d$ for some graph $\calg = (\calv, \cale)$.
We shall assume that both $\calv$ and $\cale$ are finite and adopt the convention that $n \triangleq |\calv|$ and $m \triangleq |\cale|$.
We shall identify $\calv$ with $[n]$ and $\cale$ with $[m]$.
We assume that $\calg$ is undirected, i.e. $(i, j) \in \cale \Leftrightarrow (j, i) \in \cale$ for all $i, j \in \calv$.
We assume that $\calg$ has no self-connection, i.e. $(i, i) \ne \cale$ for $i \in [n]$.
Let $M_i$ denote the $i$-th row of $M$, and $M^i$ denote the $i$-th column of $M$.

\subsubsection{Laplacian on graphs}

A real-valued function (signal) on $\calg = (\calv, \cale)$ is a mapping $\calv \mapsto \bbr$, which can be identified with a vector in $\bbr^n$ by identifying $\calv$ with $[n]$.
We assume that along with $\calg$ comes a real symmetric matrix $W$ and $W_{i j}$ is a positive number interpreted as the strength of connection between $i, j \in \calv$.
The degree of $i \in \calv$ is defined as
\[
\deg (i) \triangleq \sum_{j = 1}^n W_{i j}
\]
and the degree matrix $D$ of $\calg$ is an $n \times n$ diagonal matrix defined as
\[
D_{i j} \triangleq \delta_{i j} \deg (i)
\]
The Laplacian of $\calg$ is defined as
\[
L \triangleq D - W
\]
An alternative motivation for the definition of graph Laplacian is the following.
Suppose $u: \calv \times \bbr \mapsto \bbr$ describes the evolution of heat distribution over $\calv$.
Newton's law of cooling states that the heat transferred between $i, j \in \calv$ in an infinitestimal period of time is proportional to the difference $u_i - u_j$ when $(i, j) \in \cale$.
Consequently,
\begin{align*}
\pp{u_i}{t}
& = -c \sum_{(i, j) \in \cale} w_{i j} (u_i - u_j) \\
& = -c \left(\deg (i) u_i - \sum_{(i, j) \in \cale} w_{i j} u_j\right)
\end{align*}
Vectorizing the identity, we have
\begin{align*}
\pp{u}{t}
& = -c (D u - A u) \\
& = -c (D - A) u \\
& = -c L u
\end{align*}

\subsubsection{Convolution on graphs}

\begin{equation*}
f * g (x) \triangleq \int_{\bbr^d} f(y) (\calt_x (\calr g)) (y) dy
\end{equation*}
where $(\calr g) (\cdot) \triangleq g(-\cdot)$ is the reflection operator and $(\calt_x g) (\cdot) \triangleq g(x + \cdot)$ is the translation (by $x$) operator.
Neither $\calr$ nor $\calt_x$ makes sense when we replace $\bbr^d$ by $\calg$.
\begin{equation*}
f * g = \calf^{-1} (\calf  f \calf g)
\end{equation*}

Other definitions of Laplacian exists in the literature.
For example, the random walk Laplacian is defined as
\[
L^{RW} \triangleq I - W D^{-1}
\]
However, the random walk Laplacian is in general asymmetric and thus unsuitable for our purpose.
It follows from the spectral theorem for real symmetric matrix that $L$ admits the decomposition
\begin{equation*}
L = U^T \Lambda U
\end{equation*}
where $U \in SO(n)$, $L U_i = \Lambda_{i i} U_i$ for $i \in [n]$, $\Lambda$ is diagonal, and $\Lambda_{i i} \leq \Lambda_{j j} \forall 1 \leq i < j \leq n$.
\begin{equation*}
(U f)_i
= f^T U_i
= <f, U_i>_{\bbr^b}
\end{equation*}
which is analoguous to
\begin{equation*}
(\calf f) (n)
\triangleq \int_T e^{-i n x)} f(x) \dif x
= <e^{-i n \cdot}, f>_{L^2}
\end{equation*}
We can define Fourier transform $\calf$ on $\calg$ analoguously as
\begin{equation*}
\calf f \triangleq U f
\end{equation*}
Because $U^T U = I$ on $\bbr^n$ and $\calf^{-1} \calf = I$ on $L^2$, a natural definition for inverse Fourier transform on $\calg$ is
\begin{equation*}
\calf^{-1} \triangleq U^T
\end{equation*}

\begin{equation*}
f * g
\triangleq \calf^{-1} (\calf f \cdot \calf g)
= U^T (U f \odot U g)
\end{equation*}
where $\odot$ denotes the Hadamard product.
Analoguous to the approach to convolution on image, one of $f$ and $g$ filter and the other signal.
In the following let $x$ denote a signal and $\theta$ denote a filter.
\[
\theta * x
= U^T (U \theta \odot U x)
\]
Because $U \in SO(n)$,
\[
\theta * x
= U^T (\theta \odot U x)
= U^T (\diag (\theta) U x)
\]
(possibly with nonlinearity attached)

However, this construction has two disadvantages.
First,
Second, computing the eigen-decomposition $U^T \Lambda U$ is expensive.
One solution, proposed by \cite{defferrard2016convolutional}, is to replace $\diag (\theta)$ with a polynomial of $\Lambda$ with learnable coefficient.
Suppose $\Theta = \sum_{k = 0}^K \theta_k \Lambda^k$.
\begin{align*}
\Theta * x
& = U^T \Theta U x \\
& = U^T \left(\sum_{k = 0}^K \theta_k \Lambda^k \right) U x \\
& = \left(\sum_{k = 0}^K \theta_k U^T \Lambda^k U \right) x \\
& = \sum_{k = 0}^K \theta_k L^k x
\end{align*}
where the last identity follows from the fact that
\[
L L
= U^T \Lambda U U^T \Lambda U
= U^T \Lambda^2 U
\]
It follows from the definition of $L$ that $L$ contains $n + 2 m$ non-zero entries, which is significantly less than $n^2$ when $\calg$ is sparse, i.e. when $m / n^2 \rightarrow 0$.
An efficient implementation of $L x$, commonly referred to as Sparse Matrix-Vector multiplication (SpMV), only requires $\calo ()$ time.
The choice
\[
\Theta = 
\]
results in the popular Graph Convolutional Network (GCN) \cite{kipf2016semi}.

\subsubsection{Spectral clustering revisited}

In the following we describe a neural network architecture motivated by an improved spectral clustering algorithm.

A natural setting for theoretical analysis of spectral clustering algorithms is the Stochastic Block Model (SBM).
In the binary case, given a collection of nodes $\calv = [n]$ and a label assignment $\sigma: \calv \mapsto {\pm 1}$, the probability for two nodes $i, j \in \calv$ to be connected is
\[
\bbp [(i, j) \in \cale]
= \ind_{\sigma_i = \sigma_j} p_{in} + \ind_{\sigma_i \ne \sigma_j} p_{out}
\]
where $p_{in}$ denotes the probability of an inter-community edge and $p_{out}$ denotes the probability of an extra-community edge.
To model sparse graphs, let
\[
p_{in} = \frac{c_{in}}n \qquad
p_{out} = \frac{c_{out}}n
\]
where $c_{in}$ and $c_{out}$ are constants, and $n \rightarrow \infty$.
detectability threshold:
\[
c_{in} - c_{out}
\]
SBM can model many real world graphs, in particular, social networks.
"Birds of a feather flock together."
Most real world graphs are bounded.

It turns out that standard spectral clustering fails at the limit $n \rightarrow \infty$ even if detectable.


\section{Optimization of neural networks}

\subsection{Stochastic gradient descent}

\subsection{The challenge of saddle points}

It turns out that compared with encountering local minima, it is significantly more likely to encounter saddle points if one applies SGD to neural networks.
The following is several related results.

\bibliographystyle{unsrt}
\bibliography{written}

\end{document}
